\documentclass{article}
\usepackage[utf8]{inputenc}
\usepackage[T1]{fontenc}
\usepackage{ngerman}
\usepackage{amsmath}
\usepackage{amsfonts}
		     
		      
\title{Praktikum Multicore-Programmierung \\ Abschlussprojekt 1}
\author{Sarah Lutteropp und Johannes Sailer}

\date{\today}
% Hinweis: \title{um was auch immer es geht}, \author{wer es auch immer 
% geschrieben hat} und  \date{wann auch immer das war} k\"onnen vor 
% oder nach dem  Kommando \begin{document} stehen 
% Aber der \maketitle Befehl mu\ss{} nach dem \begin{document} Kommando stehen! 
\begin{document}

\maketitle


\begin{abstract}
Ausarbeitung für das Abschlussprojekt ``Projekt 1: Parallele Lösungsmethodik partieller Differentialgleichungen'' für das Praktikum Multicore-Programmierung im WS 15/16.
\end{abstract}

\section{Mathematischer Hintergrund}

Wir haben folgendes Setting:
\begin{itemize}
	\item $\Omega \cup \Gamma = [0,1]^2$
	\item Rand $\Gamma = \{(x,y) \in \mathbb{R}^2 : x \in \{0,1\} \text{ oder }  y \in \{0,1\} \}$
	\item $-\Delta u(x,y) = f(x,y) \quad \forall (x,y) \in (0,1)^2$
	\item $u(x,y) = 0 \quad \forall (x,y) \in [0,1]^2 \backslash (0,1)^2$
	
	\item Diskretisierungsparameter $h$
	\item $(x,y)$ sind also Koordinaten im diskretisierten Einheitsquadrat
\end{itemize}

Blabla
\subsection{Jacobi-Verfahren}
Blabla
\subsubsection{Herleitung}
Blabla
\subsubsection{Abbruchkriterium}
Blabla

\subsection{Gauß-Seidel-Verfahren}
Blabla
\subsubsection{Herleitung}
Blabla
\subsubsection{Abbruchkriterium}
Blabla

\subsection{Vergleich der Konvergenz und Stabilität beider Verfahren}
Blabla

\section{Serielle Implementierung}
Blabla
\subsection{Laufzeiten bei verschiedenen Verfeinerungen}
Blabla
\subsection{Approximationsfehler}
Blabla

\section{Parallelisierung}
Blabla
\subsection{Jacobi-Verfahren}
Blabla
\subsection{Gauß-Seidel-Verfahren}
Blabla
\subsubsection{Naiver Parallelisierungsansatz}
Blabla
\subsubsection{Erweiterter Parallelisierungsansatz}
Blabla

\section{Methodenwahl}
Blabla

\end{document}
