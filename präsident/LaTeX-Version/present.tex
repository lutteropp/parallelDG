\documentclass{beamer}
\usetheme[deutsch]{KIT}

%\usepackage{etex}
\usepackage[utf8]{inputenc}
\usepackage[T1]{fontenc}
\usepackage{babel}
\usepackage{tikz,calc,ifthen}
\usepackage{mathtools}
\usepackage[normalem]{ulem}
\usepackage{graphicx}
\usepackage{listings, caption}
\usepackage{color}
\usepackage{textcomp}
\usepackage{eurosym} % used for \euro
%\usepackage{pgfplots}
%\usepgfplotslibrary{statistics}
\usetikzlibrary{positioning,calc,arrows,shapes}
\tikzset{
  every node/.style={transform shape},
  auto,
  block/.style={align=center,rectangle,draw,minimum height=20pt,minimum width=30pt},
  >=triangle 60,
  alt/.code args={<#1>#2#3}{%
      \alt<#1>{\pgfkeysalso{#2}}{\pgfkeysalso{#3}}
  },
  beameralert/.style={alt=<#1>{color=green!80!black}{}},
  mythick/.style={line width=1.4pt}
}

\newcommand*{\maxwidthofm}[2]{\maxof{\widthof{$#1$}}{\widthof{$#2$}}}
\newcommand<>*{\robustaltm}[2]{
  \alt#3
  {\mathmakebox[\maxwidthofm{#1}{#2}]{#1}\vphantom{#1#2}}
    {\mathmakebox[\maxwidthofm{#1}{#2}]{#2}\vphantom{#1#2}}
}

\newcommand<>*{\nodealert}[1]{\only#2{\draw[overlay,mythick,color=green!80!black] (#1.north west) rectangle (#1.south east)}}

\title{Projekt 1 -- Jacobi- und Gauß-Seidel-Verfahren}
\author{Sarah Lutteropp und Johannes Sailer}
\subtitle{\insertauthor}
\institute[Lehrstuhl für Rechnerarchitektur und Parallelverarbeitung]{Lehrstuhl für Rechnerarchitektur und Parallelverarbeitung}
\date{17.02.2016}
\KITtitleimage{images/28740_the_matrix_matrix_code.jpg}

\begin{document}

\begin{frame}
    \maketitle
\end{frame}

\begin{frame}
   \frametitle{Gliederung}
   \tableofcontents
 \end{frame}

\section{Aufgabenstellung}

\begin{frame}
\frametitle{Aufgabenstellung}
TODO
\end{frame}

\section{Mathematischer Hintergrund}

\subsection*{Herleitung der Verfahren}

\begin{frame}
\frametitle{Herleitung der Verfahren}
TODO
\end{frame}

\subsection*{Abbruchkriterium}

\begin{frame}
\frametitle{Unser Abbruchkriterium}
TODO
\end{frame}

\section{Parallelisierung}

\begin{frame}
\frametitle{Parallele Ansätze -- Jacobi-Verfahren}
TODO
\end{frame}

\begin{frame}
\frametitle{Parallele Ansätze -- Gauß-Seidel-Wavefront}
TODO
\end{frame}

\begin{frame}
\frametitle{Parallele Ansätze -- Gauß-Seidel-RotSchwarz}
TODO
\end{frame}

\section{Experimentelle Auswertung}

\begin{frame}
\frametitle{Auswertung ohne Abbruchkriterium}
TODO
\end{frame}

\begin{frame}
\frametitle{Auswertung mit Abbruchkriterium}
TODO
\end{frame}

\section{Fazit}

\begin{frame}
\frametitle{Fazit}
TODO
\end{frame}

\end{document}


