\documentclass{beamer}
\usetheme[deutsch]{KIT}

%\usepackage{etex}
\usepackage[utf8]{inputenc}
\usepackage[T1]{fontenc}
\usepackage{babel}
\usepackage{tikz,calc,ifthen}
\usepackage{mathtools}
\usepackage[normalem]{ulem}
\usepackage{graphicx}
\usepackage{listings, caption}
\usepackage{color}
\usepackage{textcomp}
\usepackage{eurosym} % used for \euro
%\usepackage{pgfplots}
%\usepgfplotslibrary{statistics}
\usetikzlibrary{positioning,calc,arrows,shapes}
\tikzset{
  every node/.style={transform shape},
  auto,
  block/.style={align=center,rectangle,draw,minimum height=20pt,minimum width=30pt},
  >=triangle 60,
  alt/.code args={<#1>#2#3}{%
      \alt<#1>{\pgfkeysalso{#2}}{\pgfkeysalso{#3}}
  },
  beameralert/.style={alt=<#1>{color=green!80!black}{}},
  mythick/.style={line width=1.4pt}
}

\newcommand*{\maxwidthofm}[2]{\maxof{\widthof{$#1$}}{\widthof{$#2$}}}
\newcommand<>*{\robustaltm}[2]{
  \alt#3
  {\mathmakebox[\maxwidthofm{#1}{#2}]{#1}\vphantom{#1#2}}
    {\mathmakebox[\maxwidthofm{#1}{#2}]{#2}\vphantom{#1#2}}
}

\newcommand<>*{\nodealert}[1]{\only#2{\draw[overlay,mythick,color=green!80!black] (#1.north west) rectangle (#1.south east)}}

\title{Projekt 1 -- Jacobi- und Gauß-Seidel-Verfahren}
\author{Sarah Lutteropp und Johannes Sailer}
\subtitle{\insertauthor}
\institute[Lehrstuhl für Rechnerarchitektur und Parallelverarbeitung]{Lehrstuhl für Rechnerarchitektur und Parallelverarbeitung}
\date{17.02.2016}
\KITtitleimage{images/28740_the_matrix_matrix_code.jpg}

\begin{document}

\begin{frame}
    \maketitle
\end{frame}

\begin{frame}
   \frametitle{Gliederung}
   \tableofcontents
 \end{frame}

\section{Aufgabenstellung}

\begin{frame}
\frametitle{Aufgabenstellung}
\begin{center}
Approximation von Stoffkonzentrationen
\includegraphics[scale=0.5]{images/aufgabenstellung.png}
\\Löse $Au=b$
\end{center}
\end{frame}

\section{Mathematischer Hintergrund}

\subsection*{Herleitung der Verfahren}

\begin{frame}
\frametitle{Herleitung der Verfahren}
TODO
\end{frame}

\subsection*{Abbruchkriterium}

\begin{frame}
\frametitle{Unser Abbruchkriterium}
\begin{huge}
$$\frac{\sum_{i,j} {| u_{i,j}^{(k)} - u_{i,j}^{(k-1)} |}}{size * size} \leq \texttt{TOL}$$
\end{huge}

\begin{columns}[c]
		\column[c]{5cm}
		\vspace{0.6cm}
		\begin{block}{Vorteile}
\begin{itemize}
	\item Sprunglos
	\item Implementierung mit \texttt{\#pragma omp reduce}
\end{itemize}
\end{block}
		\column{5cm}
		\begin{block}{Nachteile}
\begin{itemize}
\item Maximum der Differenzen wäre exakter
\end{itemize}
\end{block}
	\end{columns}
\end{frame}

\begin{frame}
\frametitle{Unser Abbruchkriterium}
\begin{center}
Beide Verfahren konvergieren.
\includegraphics[scale=0.232]{images/konvergenz.png}
\end{center}
\end{frame}

\section{Parallelisierung}

\begin{frame}
\frametitle{Parallele Ansätze -- Jacobi-Verfahren}
Keine Abhängigkeiten innerhalb einer Iteration
\begin{center}
\includegraphics[scale=0.75]{images/code_jacobi.png}
\end{center}
\end{frame}

\begin{frame}
\frametitle{Parallele Ansätze -- Jacobi-Verfahren}
\begin{block}{Zusätzliche Optimierung: SSE-Vektorinstruktionen}
\begin{center}
\includegraphics[scale=0.4]{images/simd.png}
\end{center}
\end{block}
\end{frame}

\begin{frame}
\frametitle{Parallele Ansätze -- Gauß-Seidel-Wavefront}
\begin{itemize}
\item Abhängigkeiten innerhalb einer Iteration:
$$u_{i,j}^{k+1} = \frac{1}{4} u_{i,j-1}^{k+1} + u_{i-1,j}^{k+1} + u_{i,j+1}^{k} + u_{i+1,j}^{k} + h^2 f(x_i, y_j)$$
\item 1. Möglichkeit: Wavefront
\begin{center}
\includegraphics[scale=0.7]{images/wavefront.png}
\end{center}
\end{itemize}
\end{frame}

\begin{frame}
\frametitle{Parallele Ansätze -- Gauß-Seidel-Wavefront}

\begin{block}{Nachteile Wavefront}
\begin{itemize}
	\item Schlecht für Cache
	\begin{center}
	\includegraphics[scale=0.8]{images/cache_1.png} \qquad
	\includegraphics[scale=0.8]{images/cache_2.png}
	\end{center}
	\item Aufwändige Berechnung der Indizes
	\item Geringe Parallelität bei kleinen Diagonalen
	\item Allgemein großer Overhead
\end{itemize}
\end{block}

\end{frame}

\begin{frame}
\frametitle{Parallele Ansätze -- Gauß-Seidel-RotSchwarz}
TODO
\end{frame}

\section{Experimentelle Auswertung}

\begin{frame}
\frametitle{Auswertung ohne Abbruchkriterium}
TODO
\end{frame}

\begin{frame}
\frametitle{Auswertung mit Abbruchkriterium}
TODO
\end{frame}

\section{Fazit}

\begin{frame}
\frametitle{Fazit}
TODO
\end{frame}

\end{document}


